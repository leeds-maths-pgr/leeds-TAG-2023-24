% !TeX root = 000-main.tex
\documentclass[000-main.tex]{subfiles}

\begin{document}

\subsection{More Examples of Schemes}%
\label{sec:scheme-examples}

\marginnote{LDH: Include some of the notes here}

\begin{example}
	Having considered some of the easy Spec examples, let's consider something a bit more exotic, namely $\Spec \mathbb{Z}$ and $\Spec \mathbb{Z}[x]$.


	Restrict to $\Spec \mathbb{Z}$. The first thing to ask ourselves is what the prime ideals are going to be? Well in this case history comes along and makes this very easy, they are just ideals $(p)$ where $p$ is a prime, and obviously $(0)$. We note for a second that this means we have the chain of prime ideals $(0) \subset (p)$, and these are the only such chains, so the Krull dimension of $\mathbb{Z}$ is 1. We will thus want to imagine in our mind that $\Spec \mathbb{Z}$ is ``1-dimensional". We imagine a line of points $(p)$, and a generic point $(0)$ spread out amongst them.

	Now consider $\Spec \mathbb{Z}[x]$. We now also get prime ideals $(f)$ where $f$ is an irreducible polynomial, and ideals $(f, p)$ where $f$ is also irreducible mod $p$, i.e. irreducible in $\mathbb{Z}_p[x]$. We can now get chains $(0) \subset (p) \subset (f, p)$ so the Krull dimension is 2, so we want to imagine a plane. There will be a `$(p)$'-axis and an `$(f)$'-axis, with the closed points being of the form $(f, p)$. For example the $(x)$ line will intersect the lines $(2)$, $(3)$, $(5)$ just in one  point. The `line' $(x^2+1)$ will intersect $(2)$ at $(2, x+1)$ twice (as $(x+1)^2 = x^2 + 1$ mod 2) and the line $(5)$ at the two points $(5, x+2)$, $(5, x+3)$. $x^2+1$ remains irreducible over $\mathbb{Z}_3$, so the intersection $(3, x^3+1)$ is a `fat point' in some sense.
\end{example}
\end{document}

% Local Variables:
% TeX-master: "000-main.tex"
% TeX-engine: luatex
% End:
