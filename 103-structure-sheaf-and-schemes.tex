% !TeX root = 000-main.tex
\documentclass[000-main.tex]{subfiles}

\begin{document}
\subsubsection{The Structure Sheaf and Schemes}%
\label{sec:structure-sheaf}

\begin{definition}
  Let $\mathcal{O}_X$ be a sheaf on $X := \Spec A$ given by,
  \[
    \mathcal{O}_X(U) = \qty{
      s \in \Gamma\qty(U, \bigsqcup_{\mathfrak{p} \in U} A_{\mathfrak{p}})
      \middle\vert
      s\text{ satisfies~\eqref{eq:str-sheaf-cond-harts}}
    }
  \]
  where~\eqref{eq:str-sheaf-cond-harts} is the condition,
  \begin{equation} \label{eq:str-sheaf-cond-harts}
      \forall \mathfrak{p}\in U,
      \exists f_{\ast} \not\in \mathfrak{p},
      \exists V_{\ast}\in \mathcal{N}(\mathfrak{p}, U\cap D(f_{\ast})),
      \exists \alpha_{\ast} \in A_{f_\ast},
      \forall \mathfrak{q} \in V, s(\mathfrak{q}) = (\mathfrak{q}, \alpha_{\ast})
  \end{equation}
  $\mathcal{N}(p, U)$ being the set of open neighbourhoods of $p$ contained in $U$.

  We see that $\mathcal{O}_X$ is a sheaf of rings by defining $s + t$ and $st$ pointwise on fibres.
\end{definition}

The condition~\eqref{eq:str-sheaf-cond-harts} just says that the function $s$ looks locally like a quotient, mimicking the definition of a regular function on a variety (which is locally a quotient of polynomials).
This is made precise in the following manner,
\begin{proposition}
  For every $f\in A$, we have $\mathcal{O}_X(D(f)) \cong A_f$.
\end{proposition}
The proof below follows \citeauthor{hartshorneAlgebraicGeometry1977}~\cite[][71-72]{hartshorneAlgebraicGeometry1977} with some minor changes.
\begin{proof}\TOPROVE
  Fix $f\in A$ and recall that for every $\mathfrak{p}\in D(f)$ we have a natural map $\imath_{\mathfrak{p}}: A_f \to A_\mathfrak{p}$.
  Thus we define,
  \begin{displaymath}
    \begin{gathered}
      \psi: A_f \to \Gamma\qty(D(f), \coprod_{\mathfrak{p}\in D(f)}A_\mathfrak{p})\\
      \alpha \mapsto (\mathfrak{p}\mapsto (\mathfrak{p}, \imath_{\mathfrak{p}}(\alpha)))
    \end{gathered}
  \end{displaymath}
  Note that for any $\alpha\in A_f$ and any $\mathfrak{p}\in D(f)$, we can just take $(f_\ast, V_\ast, \alpha_\ast) := (f, D(f), \alpha)$ and we satisfy~\eqref{eq:str-sheaf-cond-harts}.
  Thus $\psi$ is a well defined map $\psi: A_f \to \mathcal{O}_X(D(f))$, the fact that it is a homomorphism of rings follows immediately from $\imath_{\mathfrak{p}}$ being a homomorphism of rings.

  We first show that $\psi$ is injective.
  Suppose $\psi(a/f^n) = \psi(b/f^m)$, then for any $\mathfrak{p}\in D(f)$, we must have $\imath_\mathfrak{p}(a/f^n) = \imath_\mathfrak{p}(b/f^m)$, i.e.\ letting $\mathfrak{a := }\Ann(af^m - bf^n)$, $(A\setminus \mathfrak{p})\cap \mathfrak{a} = \mathfrak{a}\setminus\mathfrak{p}\neq\varnothing$.
  Thus,
  \begin{align*}
    \forall\mathfrak{p}\in D(f), \mathfrak{a} \not\subset \mathfrak{p}
    &\implies \forall \mathfrak{p}\in D(f), \mathfrak{p}\not\in V(\mathfrak{a})\\
    &\implies D(f)\cap V(\mathfrak{a}) = \varnothing\\
    &\implies V(f) \supset V(\mathfrak{a})\\
    &\implies f \in \sqrt{\mathfrak{a}}\\
    &\implies \exists l\in\Zpl, f^l\in \mathfrak{a}\\
    &\implies \exists l\in\Zpl, f^l(af^m - bf^n) = 0\\
    &\implies a/f^n = b/f^m
  \end{align*}
  and we are done.

  To show surjectivity, suppose $s\in \mathcal{O}_X(D(f))$.
  By~\eqref{eq:str-sheaf-cond-harts}, for every $\mathfrak{p}\in D(f)$, there exists:
  \begin{itemize}
    \item $f_\mathfrak{p}\not\in \mathfrak{p}$,
    \item $V_\mathfrak{p}\in \mathcal{N}(p, D(f)\cap D(f_\mathfrak{p}))$
  \end{itemize}
  such that $\exists \alpha_\mathfrak{p}\in A_{f_\mathfrak{p}}, \forall \mathfrak{q}\in V_\mathfrak{p}, s(\mathfrak{q}) = \alpha_\mathfrak{p}$.

  Assuming WLOG $V_\mathfrak{p} = D(h_\mathfrak{p})$, we see that $D(h_\mathfrak{p}) \subset D(f)\cap D(f_\mathfrak{p}) \implies h_\mathfrak{p}\in \sqrt{(f_\mathfrak{p})}$
\end{proof}

Moreover, the structure sheaf has the following important property,
\begin{lemma}
  The stalk $\mathcal{O}_{\mathfrak{p}}$ is isomorphic to $A_{\mathfrak{p}}$, not just as a ring but as a local ring.
\end{lemma}
\begin{proof}\TOPROVE
  Let $\mathfrak{p}\in\Spec A$, we then have $\mathcal{O}_{X,\mathfrak{p}} := \varinjlim_{U\ni\mathfrak{p}}\mathcal{O}_X(U) = \varinjlim_{D(f)\ni\mathfrak{p}}\mathcal{O}_X(D(f))$ which is isomorphic to $\varinjlim_{f\in A\setminus\mathfrak{p}}A_f$.
\end{proof}

We call the pair $(\Spec A, \mathcal{O})$ the \emph{spectrum} of $A$.

\begin{definition}
  We shall call a pair $(X, \mathcal{O}_X)$, where $X$ is a topological space and $\mathcal{O}_X$ is a sheaf of rings over it, a \emph{ringed space}.
  Define a morphism of ringed spaces to be a pair $(f : X \to Y, f^\sharp : \mathcal{O}_Y \to f_\ast \mathcal{O}_X)$, where $f, f^\sharp$ are morphisms in the corresponding categories.

  We shall say that $(X, \mathcal{O}_X)$ is moreover a \emph{locally ringed space} if the stalk $\mathcal{O}_{X, P}$ at $P \in X$ is a local ring and $f^\sharp_P : \mathcal{O}_{Y, f(P)} \to \mathcal{O}_{X, P}$ is a local ring hom.
\end{definition}

\begin{lemma}
  $(\Spec A, \mathcal{O})$ is a locally ringed space, and the assignment $A \mapsto (\Spec A , \mathcal{O})$ is a contravariant functor.
\end{lemma}

These brief preliminaries have led up to the definition we wanted in this section.

\begin{definition}
  An \emph{affine scheme} is a locally ringed space $(X, \mathcal{O}_X)$ which is isomorphic to $(\Spec A, \mathcal{O})$ for some $A$.
  A \emph{scheme} is a locally ringed space such that $\forall p \in X$, $\exists U \in \mathcal{N}(p, X)$ such that $(U, \left . \mathcal{O}_X \right \rvert_{U})$ is an affine scheme.
\end{definition}

\begin{example}\label{ex: gluing schemes}
  One way to construct schemes which will allow us to get non-affine examples is by gluing. Recall that one of the constructions we had for sheaves was that, given sheaves $\mathcal{F}_i$ on $U_i \subset X$ open such that $\qty{U_i}$ is a cover of $X$, and moreover given isomorphisms $\phi_{ij} : \mathcal{F}_{i} |_{U_i \cap U_j} \to \mathcal{F}_j |_{U_i \cap U_j}$ such that the $\phi_{ij}$ satisfy cocycle conditions, then there exists a unique sheaf $\mathcal{F}$ on $X$ with isomorphisms $\psi_i : \mathcal{F} |_{U_i} \to \mathcal{F}_i$ related by the $\phi_{ij}$. 
  
  We will have a similar construction for schemes. Namely, given $X_i$ schemes and $U_{ij} \subset X_i$ with isomorphisms $\phi_{ij} : U_{ij} \to U_{ji}$ satisfying cocycle conditions, there exists a unique sheaf $X$ with morphisms $\psi_i : X_i \to X$ such that $\qty{\psi_i(X_i)}$ is an open cover of $X$ and the $\psi_i$ are related by the $\phi_{ij}$. (This is Hartshorne Exercise II.2.12).  
\end{example}

One important example of a scheme that is not affine will be $\Proj$, defined as follows: given a graded ring $S$ (with $S_+ = \bigoplus_{d > 0 } S_d$) the underlying topological space is
\[
  \Proj S = \left \lbrace \mathfrak{p} \text{ homogeneous prime ideal} \, | \, S_+ \subsetneq \mathfrak{p} \right \rbrace ,
\]
with topology and sheaf defined as before, but now with the sheaf given by maps into $\bigcup_{\mathfrak{p} \in U} S_{(\mathfrak{p})}$, where $S_{(\mathfrak{p})}$ is the ring of degree-0 elements of $T^{-1}S$ for $T$ the homogeneous elements not in $\mathfrak{p}$, and we require the sections to be locally a ration of element of the same fixed degree (equivalent to saying the map is into the degree 0 part).

\marginnote{GC: Discussion about why this is locally affine required.}
\end{document}

% Local Variables:
% TeX-master: "000-main.tex"
% TeX-engine: luatex
% End:
