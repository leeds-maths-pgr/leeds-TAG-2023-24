\documentclass[000-main.tex]{subfiles}

\begin{document}

\subsection{$\Spec$ and Schemes}%
\label{sec:schemes}

Suppose $A$ is a (commutative) ring, and recall that an ideal $\mathfrak{p}\triangleleft A$ is \emph{prime} if $ab\in\mathfrak{p}\implies a\in \mathfrak{p}$ or $b\in\mathfrak{p}$.
As a matter of convention, we will assume that a prime ideal is necessarily not equal to the entire ring.
\begin{definition}
  $\Spec A$ is the set,
  \begin{displaymath}
	\Spec A := \left\{ \mathfrak{p}\subset A \mid \mathfrak{p}\text{ is a prime ideal of } A\right\}
  \end{displaymath}

  The \emph{vanishing} of an ideal $\mathfrak{a}\trianglelefteq A$ is the set,
  \begin{displaymath}
	V(\mathfrak{a}) := \left\{ \mathfrak{p}\in\Spec A \mid \mathfrak{p} \supseteq \mathfrak{a} \right\}
  \end{displaymath}
\end{definition}
We define a topology on $\Spec A$ by letting the $V(\mathfrak{a})$ be the closed sets. This is analogous to the Zariski topology on a variety.
For a general $S\subset A$, we define $V(S)$ to be the vanishing of the ideal generated by $S$.
If $S = \left\{ f \right\}$, we will often just write $V(f)$.

\begin{lemma}[\citeauthor{hartshorneAlgebraicGeometry1977}~{\cite[][70]{hartshorneAlgebraicGeometry1977}} Lemma II.2.1]
	The topology defined really is a topology. 
\end{lemma} 

\begin{example}[Fields and Polynomial Rings]
  Let $k$ be some field.
  \begin{itemize}
    \item If $A = k$, the only prime ideal is $(0)$, so $\Spec k$ is a point. Note that $k$ has Krull dimension 0.
    \item If $A = k[x]$, every prime ideal is either the zero ideal, or it is generated by some irreducible polynomial $f$.
    Moreover, when $k$ is algebraically closed the only irreducible polynomials are of the form $f = (x-a)$ for $a \in k$.

    Clearly $(0)$ is never in the closure of any ideal (except itself), and $V(x-a) = \left\{(x-a)\right\}$, so each $(x-a)$ is a closed point.\marginnote{GC to LDH: What do you mean by closure here?}%
    Thus the closed points are in bijection with the underlying field and we have one \emph{generic} point $(0)$ for which $V(0) = \Spec A$.
    We again note that $k[x]$ has Krull dimension 1.

    \item Finally, consider $A = k[x,y]$ where $k$ is again algebraically closed.
    The points of $\Spec A$ are:
    \begin{itemize}
			\item $(0)$,
			\item  $(f)$ for $f \in k[x, y]$ irreducible,
			\item  and $(x-a, y-b)$ for $a,b \in k$.
    \end{itemize}
    The closed points are precisely the $(x-a, y-b)$, and given $f$ such that $f(a,b)=0$, we see that $V(f) \cap V((x-a, y-b)) = V((x-a, y-b))$.
    Therefore we have generic points distributed along the zero locus of any irreducible $f$, and we note that $k[x, y]$ has Krull dimension 2.
  \end{itemize}
\end{example}

\begin{remark}
  Note that in the examples above, we have $\dim A[x] = \dim A + 1$.
  For a general ring $R$, it is easy to see that $\dim R[x] \geq \dim R + 1$, in fact we have equality if $R$ is Noetherian.\marginnote{GC: Converse?}
  This is the last exercise in \citeauthor{atiyahIntroductionCommutativeAlgebra2000}~\cite{atiyahIntroductionCommutativeAlgebra2000}.
\end{remark}

It is also useful to define,
\begin{displaymath}
  D(\cdot) := \Spec A\setminus V(\cdot)
\end{displaymath}
Explicitly, $D(S) = \left\{ \mathfrak{p}\in\Spec A \mid \mathfrak{p}\not\supseteq S \right\}$.
A convenient basis of open sets is the basis of \emph{distinguished open sets}, $\left\{ D(f) \mid f\in A \right\}$.
\marginnote{GC: Intuition about this basis? Complement of closed sets, non-vanishing of polynomials.}

We will now define a sheaf over $\Spec A$ such that $\Spec A$ is a \emph{locally ringed space}.
However, it is useful to recall some basic ring theory before proceeding.
All of these facts can be found in~\cite[][81-86]{altmanTermCommutativeAlgebra2013} or~\cite[][36-49]{atiyahIntroductionCommutativeAlgebra2000}.

\paragraph{Localisation}

Recall that $S\subset A$ is \emph{multiplicative} if $a, b\in S \implies ab\in S$.
The \emph{localisation of $A$ away from $S$} is the ring consisting of pairs,
\begin{displaymath}
  S^{-1}A := \left\{ s^{-1}a \mid s\in S, a\in A \right\}/\sim
\end{displaymath}
where,
\begin{displaymath}
  s^{-1}a \sim t^{-1}b \iff
  \exists u\in S^{-1}, u(at - bs) = 0
\end{displaymath}

There are two particularly important examples of localisations:
\begin{itemize}
  \item localising away from $\left\{a, a^2, \ldots \right\}$ for $a\in A$, and
  \item localising away from $A\setminus\mathfrak{p}$ for some prime ideal $\mathfrak{p}\subset A$.
\end{itemize}
we call the first \emph{localising away from $a$} and the second \emph{localising at $\mathfrak{p}$}.

Confusingly, these rings are commonly written $A_a$ and $A_{\mathfrak{p}}$ respectively even though the subscript is playing a very different role in each case.
Even more frustratingly, the usage of localisation \emph{at} vs.\ localisation \emph{away from} is also subject to convention.
GC suggests using \emph{at} only when we are localising with respect to the complement of a prime ideal, and \emph{away from} for anything else.

In the particular case of localising at primes, $A_\mathfrak{p}$ is an example of a \emph{local ring}, i.e.\ one with a unique maximal ideal, $\mathfrak{p}$.
Moreover, given a homomorphism of rings, $\phi : A \to B$, and $\mathfrak{q}\subset B$ prime, we have a homomorphism of local rings $\phi_{\mathfrak{q}} : A_{\phi^{-1}(\mathfrak{q})} \to B_{\mathfrak{q}}$.

\paragraph{The Structure Sheaf}

We can now construct the structure sheaf on $\Spec A$.
\begin{definition}
  For a ring $A$, consider the disjoint union $A_{\bullet} := \bigsqcup_{\mathfrak{p}\in\Spec A}A_\mathfrak{p}$.
  Define the presheaf of sections to be,
  \begin{displaymath}
    \Gamma(U, A_\bullet) := \left\{ s: U \to A_{\bullet} \mid \forall\mathfrak{p}\in U, s(\mathfrak{p}) = (\mathfrak{p}, \cdot) \right\}
  \end{displaymath}
  for $U\subset\Spec A$ open.
  Moreover, let $\mathcal{O}(u)$ be the sub-presheaf of $s\in \Gamma(U, A_\bullet)$ which satisfy the condition,
  \begin{equation}\label{eqn:structure-sheaf-cond}
      \forall \mathfrak{p}\in U,
      \exists V\subset U : \mathfrak{p}\in V,
      \exists a, f\in A, \text{ s.t.\ }
      \forall \mathfrak{q}\in V,
      f\in \mathfrak{q} \text{ and }
      s(\mathfrak{q}) = f^{-1}a
      \tag{$*$}
  \end{equation}
  We will ultimately show that $\mathcal{O}$ is in fact a sheaf which we call the \emph{structure sheaf} of $\Spec A$.
  We call the pair $(\Spec A, \mathcal{O})$ the \emph{(prime) spectrum} of $A$.
\end{definition}

\begin{remark}
  While~\eqref{eqn:structure-sheaf-cond} looks intimidating, it just says that the function $s$ looks locally like a quotient, mimicking the definition of a regular function on a variety (which is locally a quotient of polynomials).

  Indeed, the fact that $\mathfrak{p}\in D(f) \implies \left\{ f, f^2, \ldots \right\}\subset A\setminus\mathfrak{p}$ yields a natural map $A_f \to A_\mathfrak{p}, c\mapsto c/1$ for any $\mathfrak{p}\in D(f)$.
  With this, we construct the sub-presheaf,
  \begin{displaymath}
    \underline{A_f}(U) := \left\{
      s: \mathfrak{p}\mapsto (\mathfrak{p}, c/1) \mid
      c \in A_f
    \right\} \subset \Gamma(U, A_{\bullet})
  \end{displaymath}
  for some $f\in A$, $U\subset D(f)$.
  We can then rewrite the definition of the structure sheaf as,
  \begin{equation}\label{eq:structure-sheaf-cond-alt}
    \mathcal{O}(U) = \qty{
      s\in\Gamma(U, A_{\bullet}) \middle\vert
      \forall p\in U,
      \exists f\in A,
      \exists V : p\in V\subset U\cap D(f),
      \res_{VU}(s) \in \underline{A_f}(V)
    }\tag{**}
  \end{equation}
  \marginnote{GC: Kinda neat consequence, though not sure if useful.}
\end{remark}

\begin{lemma}
  For every $\mathfrak{p}\in\Spec A$, we have an isomorphism of local rings, $\mathcal{O}_{\mathfrak{p}} \cong A_\mathfrak{p}$.
\end{lemma}

\begin{definition}
  If $X$ is a topological space and $\mathcal{O}_X$ is a sheaf of rings, we call the pair $(X, \mathcal{O}_X)$ a \emph{ringed space}.
  A morphism of ringed spaces is a pair $(f : X \to Y, f^\sharp : \mathcal{O}_Y \to f_\ast \mathcal{O}_X)$, where $f, f^\sharp$ are morphisms in the corresponding categories.
	
	We say that $(X, \mathcal{O}_X)$ is a \emph{locally ringed space} if the stalk $\mathcal{O}_{X, P}$ at $P \in X$ is a local ring.
  A morphism of locally ringed spaces is a morphism of ringed spaces where $f^\sharp_P : \mathcal{O}_{Y, f(P)} \to \mathcal{O}_{X, P}$ is also a homomorphism of local rings.
\end{definition}

\begin{example}
  $(\Spec A, \mathcal{O})$ is a locally ringed space.
  Moreover, we have a contravariant functor,
  \begin{displaymath}
    \begin{gathered}
      \mathbf{CRng}^{\opp} \to \mathbf{LRSp}\\
      A \mapsto (\Spec A , \mathcal{O})
    \end{gathered}
  \end{displaymath}
\end{example}

With this, we can finally define an affine scheme,
\begin{definition}
    An \emph{affine scheme} is a locally ringed space $(X, \mathcal{O}_X)$ which is isomorphic to $(\Spec A, \mathcal{O})$ for some $A$.
    A \emph{scheme} is a locally ringed space such that $\forall p \in X$, $\exists U \subseteq X$ open containing $p$ such that $(U, \left . \mathcal{O}_X \right \rvert_{U})$ is an affine scheme.
\end{definition}

\begin{example}
	\hl{Give a non-proj example of something which isn't an affine scheme.} 
\end{example}

One important example of a scheme that is not affine will be $\Proj$, defined as follows: given a graded ring $S$ (with $S_+ = \bigoplus_{d > 0 } S_d$) the underlying topological space is 
\[
\Proj S = \left \lbrace \mathfrak{p} \text{ homogeneous prime ideal} \, | \, S_+ \subsetneq \mathfrak{p} \right \rbrace ,  
\] 
with topology and sheaf defined as before, but now with the sheaf given by maps into $\bigcup_{\mathfrak{p} \in U} S_{(\mathfrak{p})}$, where $S_{(\mathfrak{p})}$ is the ring of degree-0 elements of $T^{-1}S$ for $T$ the homogeneous elements not in $\mathfrak{p}$, and we require the sections to be locally a ration of element of the same fixed degree (equivalent to saying the map is into the degree 0 part). 

\hl{Discussion about why this is locally affine required, Gautam has notes?} 

\end{document}

% Local Variables:
% TeX-master: "000-main.tex"
% TeX-engine: luatex
% End:
